\documentclass{amsart}
\usepackage{amssymb}
\usepackage{graphics}
\usepackage{graphicx}
\usepackage{verbatim}
\usepackage{amssymb, epsfig}
\usepackage[leqno]{amsmath}
\usepackage{eurosym}
\usepackage{listings}

\def\P{\mathbb{P}}
\def\Q{\mathbb{Q}}
\def\E{\mathbb{E}}

\newtheorem{theorem}{Theorem}[section]
\newtheorem{lemma}[theorem]{Lemma}
\newtheorem{proposition}[theorem]{Proposition}
\newtheorem{example}[theorem]{Example}
\newtheorem{definition}[theorem]{Definition}
\newtheorem{corollary}[theorem]{Corollary}
\newtheorem{remark}[theorem]{Remark}
\newtheorem{assumption}[theorem]{Assumption}

\textheight=9.5 in
\textwidth=6.5in
\topmargin=-.0in
\hoffset=-.75in

\title{Hedging Project}
%\author{Steve Taylor}
%\email{steve98654@gmail.com}

\begin{document}

\maketitle

Hedging is an important aspect of portfolio management.  Portfolio 
managers are often required to stay within risk limits; for example, they cannot 
let their volatility be to high or value-at-risk so low.  Also, there are times 
when market conditions lead them to reduce the risk of their portfolio.  
Hedging is the primary mechanism through which they carry out these tasks.  
Below, we give a mathematical statement of the hedging problem, ideas on how 
to start this project, and longer term opened ended questions that you 
can investigate if interested.

\section{Statement of the Hedging Problem}
%
Consider a portfolio that has daily prices $p_i$ for $i=1,\ldots,n$. Suppose you have 
selected a set of $m$ hedging securities with prices $p^j_i$ for $j=1,\ldots,m$, 
and consider the hedged portfolio whose daily price is 
%
\begin{equation}
    h_i = p_i + \sum_{j=1}^m c_j p^j_i,
\end{equation}
%
where here $c_j$ are weights defining the amount of hedging securities that 
we will hold.  Hedging is the process of selecting the hedging securities 
and associated weights $c_j$ in order to minimize a given risk metric of interest, 
e.g. the volatility of the portfolio, the VaR, CVaR, etc.

\section{Project Ideas}
\textbf{Getting Started}:

-- Start by implementing the minimum variance hedge (see attached slides for a 
definition) for a single stock (say IBM) and a market ETF (say the SPY ETF).  You 
should be able to get end of day data from google finance and using the pandas 
datareader. 

-- After designing an initial hedging strategy, develop analytics to examine how 
   significantly your hedged portfolio reduces or increases several standard
   risk and performance metrics of the original portfolio including the 
   volatility, return, Sharpe Ratio, maximum drawdown, value-at-risk, 
   beta of the portfolio to the market, etc.

-- Investigate other hedging strategies besides the minimum variance hedge.  
   For example, how can you try to minimize the variance of the 
   return distribution, VaR, CVaR?  This will lead into optimization methods 
   many of which are available in scipy.

\textbf{Further Ideas}:

-- Construct long/short portfolios of stocks in a given sector, e.g. financials, biotech, 
   as well as a set of hedging securities for this sector.  How should you adapt 
   your single stock/hedging security approach to this case? 

-- Consider a fixed income portfolio that consists of corporate bonds.  What are the 
   appropriate hedging securities for this portfolio? Can you get as significant risk 
   reductions as you saw in the equity case in this example? 

-- It is often said that gold is a hedge against inflation.  Can you quantify this? 
   For example, assume that inflation (year on year CPI) is tradeable and try to 
   hedge it with gold futures.  Can you get as good of hedging performance as in the 
   equity examples that you considered?  Can you find other futures, bonds, or stocks 
   that are better hedges for inflation than gold? 

\end{document}
